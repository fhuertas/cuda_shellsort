\section{Conclusi�n}



Los datos obtenidos en las ejecuciones de las diferentes implementaciones, sin ser concluyentes,
debido a la potencia de los dispositivos gr�ficos, hacen pensar las implementaciones paralelas
de este algoritmo pueden mejorar los tiempos de ordenaci�n con respecto a la ejecuci�n del mismo
en una CPU. Sin embargo esta mejora no es tan significativa como los resultados obtenidos con otros
algoritmos. 

Entre las causas se encuentra la deslocalizaci�n de los datos que gestiona cada hilo y la
diferencia de cantidad de datos que manejan cada subconjunto en las diferentes fases, sobre todo la
primera y la �ltima. 

Tambi�n se puede observar como, el algoritmo ofrece mejores resultados cuantos m�s datos gestiona. 

\section{Trabajo futuro} 

Entre las mejoras que se pueden realizar a la implementaci�n paralela plateada se encuentra realizar
una implementaci�n dependiente de los incrementos y diferentes implementaciones para cada
incremento. Al existir pocas fases, para un mill�n de elementos, 17 fases, la ganancia que se
obtendr� con esto, sera mucho mayor. Adem�s, esta implementaci�n permitir� que al copia en local de
los datos que cada hilo ordena, sea mucho m�s eficiente. 
