\section{Introducci�n}

\subsection{Potencia en unidades de procesamiento gr�fico, procesamiento paralelo}

Durante la �ltima d�cada, las \acp{GPU} han sufrido una gran evoluci�n ofreciendo una capacidad de
procesamiento muy superior a las \acp{CPU}. Esta superioridad esta basada en la peculiaridad de las
operaciones de procesamiento de im�genes, que se consiste, en muchos
casos, en realizar la misma operaci�n para cada uno de los puntos de una imagen. 

\newFig{arquitectura}{Diferencias arquitect�nicas entre una \ac{CPU} y una \ac{GPU}}

Aprovechando esta caracter�sticas, las \acp{GPU} basan su capacidad de computo en la utilizaci�n de
gran cantidad de n�cleos especializados que trabajan en paralelo. La arquitectura de estos n�cleos
es mucho m�s sencilla que las que tiene una \acp{CPU} como muestra la \refFig{arquitectura}. Esta
arquitectura, por tanto, no es apta para programaci�n general.

\subsection{Unidades de procesamiento gr�fico y programaci�n general}

La mejora de prestaciones de las \acp{GPU} ha aumentando el inter�s para el uso de
estos dispositivos en otro tipo de programas. Este hecho ha producido la aparici�n de
nuevos lenguajes de programaci�n como CUDA y OpenCL que faciliten el uso de los dispositivos. 

Actualmente est�n apareciendo implementaciones de algoritmos adaptados para su ejecuci�n paralela
que aprovechen la potencia de estos dispositivos gr�ficos. Estos algoritmos se caracterizan por
ser altamente paralelizables pudiendo aprovechar al m�ximo la potencia de calcula de las \acp{GPU}.


