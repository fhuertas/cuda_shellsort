
%% bare_jrnl.tex
%% V1.3
%% 2007/01/11
%% by Michael Shell
%% see http://www.michaelshell.org/
%% for current contact information.
%%
%% This is a skeleton file demonstrating the use of IEEEtran.cls
%% (requires IEEEtran.cls version 1.7 or later) with an IEEE journal paper.
%%
%% Support sites:
%% http://www.michaelshell.org/tex/ieeetran/
%% http://www.ctan.org/tex-archive/macros/latex/contrib/IEEEtran/
%% and
%% http://www.ieee.org/



% *** Authors should verify (and, if needed, correct) their LaTeX system  ***
% *** with the testflow diagnostic prior to trusting their LaTeX platform ***
% *** with production work. IEEE's font choices can trigger bugs that do  ***
% *** not appear when using other class files.                            ***
% The testflow support page is at:
% http://www.michaelshell.org/tex/testflow/


%%*************************************************************************
%% Legal Notice:
%% This code is offered as-is without any warranty either expressed or
%% implied; without even the implied warranty of MERCHANTABILITY or
%% FITNESS FOR A PARTICULAR PURPOSE! 
%% User assumes all risk.
%% In no event shall IEEE or any contributor to this code be liable for
%% any damages or losses, including, but not limited to, incidental,
%% consequential, or any other damages, resulting from the use or misuse
%% of any information contained here.
%%
%% All comments are the opinions of their respective authors and are not
%% necessarily endorsed by the IEEE.
%%
%% This work is distributed under the LaTeX Project Public License (LPPL)
%% ( http://www.latex-project.org/ ) version 1.3, and may be freely used,
%% distributed and modified. A copy of the LPPL, version 1.3, is included
%% in the base LaTeX documentation of all distributions of LaTeX released
%% 2003/12/01 or later.
%% Retain all contribution notices and credits.
%% ** Modified files should be clearly indicated as such, including  **
%% ** renaming them and changing author support contact information. **
%%
%% File list of work: IEEEtran.cls, IEEEtran_HOWTO.pdf, bare_adv.tex,
%%                    bare_conf.tex, bare_jrnl.tex, bare_jrnl_compsoc.tex
%%*************************************************************************

% Note that the a4paper option is mainly intended so that authors in
% countries using A4 can easily print to A4 and see how their papers will
% look in print - the typesetting of the document will not typically be
% affected with changes in paper size (but the bottom and side margins will).
% Use the testflow package mentioned above to verify correct handling of
% both paper sizes by the user's LaTeX system.
%
% Also note that the "draftcls" or "draftclsnofoot", not "draft", option
% should be used if it is desired that the figures are to be displayed in
% draft mode.
%
\documentclass[journal]{IEEEtran}
\usepackage[latin1]{inputenc}

\usepackage{ifpdf}
% Heiko Oberdiek's ifpdf.sty is very useful if you need conditional
% compilation based on whether the output is pdf or dvi.
% usage:
% \ifpdf
%   % pdf code
% \else
%   % dvi code
% \fi
% Esto esta as� por problemas con el ieeetran y sus mayusculas
\usepackage[spanish,english]{babel}
\selectlanguage{spanish}
\usepackage[T1]{fontenc}
\usepackage{lmodern}
\usepackage{url}
\usepackage{graphicx}
\usepackage{subfig}

% \usepackage{hyphenat}
% \RequirePackage[pdfborder=0,colorlinks,hyperindex,pdfpagelabels]{hyperref}
% Paketes descargados
\usepackage[nolist,nohyperlinks]{packages/acronym}
%----------------------------------------------------------------
%
%                          guionado.tex
%
%----------------------------------------------------------------
%
% Fichero con algunas divisiones de palabras que LaTeX no
% hace correctamente si no se le da alguna ayuda.
%
%----------------------------------------------------------------

\hyphenation{
% a
abs-trac-to
abs-trac-tos
abs-trac-ta
abs-trac-tas
ac-tua-do-res
a-gra-de-ci-mien-tos
ana-li-za-dor
an-te-rio-res
an-te-rior-men-te
apa-rien-cia
a-pro-pia-do
a-pro-pia-dos
a-pro-pia-da
a-pro-pia-das
a-pro-ve-cha-mien-to
a-que-llo
a-que-llos
a-que-lla
a-que-llas
a-sig-na-tu-ra
a-sig-na-tu-ras
a-so-cia-da
a-so-cia-das
a-so-cia-do
a-so-cia-dos
au-to-ma-ti-za-do
% b
batch
bi-blio-gra-f�a
bi-blio-gr�-fi-cas
bien
bo-rra-dor
boo-l-ean-expr
% c
ca-be-ce-ra
call-me-thod-ins-truc-tion
cas-te-lla-no
cir-cuns-tan-cia
cir-cuns-tan-cias
co-he-ren-te
co-he-ren-tes
co-he-ren-cia
co-li-bri
co-men-ta-rio
co-mer-cia-les
co-no-ci-mien-to
cons-cien-te
con-si-de-ra-ba
con-si-de-ra-mos
con-si-de-rar-se
cons-tan-te
cons-trucci�n
cons-tru-ye
cons-tru-ir-se
con-tro-le
co-rrec-ta-men-te
co-rres-pon-den
co-rres-pon-dien-te
co-rres-pon-dien-tes
co-ti-dia-na
co-ti-dia-no
crean
cris-ta-li-zan
cu-rri-cu-la
cu-rri-cu-lum
cu-rri-cu-lar
cu-rri-cu-la-res
% d
de-di-ca-do
de-di-ca-dos
de-di-ca-da
de-di-ca-das
de-rro-te-ro
de-rro-te-ros
de-sa-rro-llo
de-sa-rro-llos
de-sa-rro-lla-do
de-sa-rro-lla-dos
de-sa-rro-lla-da
de-sa-rro-lla-das
de-sa-rro-lla-dor
de-sa-rro-llar
des-cri-bi-re-mos
des-crip-ci�n
des-crip-cio-nes
des-cri-to
des-pu�s
de-ta-lla-do
de-ta-lla-dos
de-ta-lla-da
de-ta-lla-das
di-a-gra-ma
di-a-gra-mas
di-se-�os
dis-po-ner
dis-po-ni-bi-li-dad
do-cu-men-ta-da
do-cu-men-to
do-cu-men-tos
% e
edi-ta-do
e-du-ca-ti-vo
e-du-ca-ti-vos
e-du-ca-ti-va
e-du-ca-ti-vas
e-la-bo-ra-do
e-la-bo-ra-dos
e-la-bo-ra-da
e-la-bo-ra-das
es-co-llo
es-co-llos
es-tu-dia-do
es-tu-dia-dos
es-tu-dia-da
es-tu-dia-das
es-tu-dian-te
e-va-lua-cio-nes
e-va-lua-do-res
exis-ten-tes
exhaus-ti-va
ex-pe-rien-cia
ex-pe-rien-cias
% f
for-ma-li-za-do
% g
ge-ne-ra-ci�n
ge-ne-ra-dor
ge-ne-ra-do-res
ge-ne-ran
% h
he-rra-mien-ta
he-rra-mien-tas
% i
i-dio-ma
i-dio-mas
im-pres-cin-di-ble
im-pres-cin-di-bles
in-de-xa-do
in-de-xa-dos
in-de-xa-da
in-de-xa-das
in-di-vi-dual
in-fe-ren-cia
in-fe-ren-cias
in-for-ma-ti-ca
in-gre-dien-te
in-gre-dien-tes
in-me-dia-ta-men-te
ins-ta-la-do
ins-tan-cias
% j
% k
% l
len-gua-je
li-be-ra-to-rio
li-be-ra-to-rios
li-be-ra-to-ria
li-be-ra-to-rias
li-mi-ta-do
li-te-ra-rio
li-te-ra-rios
li-te-ra-ria
li-te-ra-rias
lo-tes
% m
ma-ne-ra
ma-nual
mas-que-ra-de
ma-yor
me-mo-ria
mi-nis-te-rio
mi-nis-te-rios
mo-de-lo
mo-de-los
mo-de-la-do
mo-du-la-ri-dad
mo-vi-mien-to
% n
na-tu-ral
ni-vel
nues-tro
% o
obs-tan-te
o-rien-ta-do
o-rien-ta-dos
o-rien-ta-da
o-rien-ta-das
% p
pa-ra-le-lo
pa-ra-le-la
par-ti-cu-lar
par-ti-cu-lar-men-te
pe-da-g�-gi-ca
pe-da-g�-gi-cas
pe-da-g�-gi-co
pe-da-g�-gi-cos
pe-rio-di-ci-dad
per-so-na-je
plan-te-a-mien-to
plan-te-a-mien-tos
po-si-ci�n
pre-fe-ren-cia
pre-fe-ren-cias
pres-cin-di-ble
pres-cin-di-bles
pri-me-ra
pro-ble-ma
pro-ble-mas
pr�-xi-mo
pu-bli-ca-cio-nes
pu-bli-ca-do
% q
% r
r�-pi-da
r�-pi-do
ra-zo-na-mien-to
ra-zo-na-mien-tos
re-a-li-zan-do
re-fe-ren-cia
re-fe-ren-cias
re-fe-ren-cia-da
re-fe-ren-cian
re-le-van-tes
re-pre-sen-ta-do
re-pre-sen-ta-dos
re-pre-sen-ta-da
re-pre-sen-ta-das
re-pre-sen-tar-lo
re-qui-si-to
re-qui-si-tos
res-pon-der
res-pon-sa-ble
% s
se-pa-ra-do
si-guien-do
si-guien-te
si-guien-tes
si-guie-ron
si-mi-lar
si-mi-la-res
si-tua-ci�n
% t
tem-pe-ra-ments
te-ner
trans-fe-ren-cia
trans-fe-ren-cias
% u
u-sua-rio
Unreal-Ed
% v
va-lor
va-lo-res
va-rian-te
ver-da-de-ro
ver-da-de-ros
ver-da-de-ra
ver-da-de-ras
ver-da-de-ra-men-te
ve-ri-fi-ca
% w
% x
% y
% z
}
% Variable local para emacs, para que encuentre el fichero
% maestro de compilaci�n
%%%
%%% Local Variables:
%%% mode: latex
%%% TeX-master: "./Tesis.tex"
%%% End:



%
% If IEEEtran.cls has not been installed into the LaTeX system files,
% manually specify the path to it like:
% \documentclass[journal]{../sty/IEEEtran}

\newcommand{\labelEq}[1]{\label{eq:#1}}
\newcommand{\labelFig}[1]{\label{fig:#1}}
\newcommand{\labelTab}[1]{\label{tab:#1}}
\newcommand{\labelSec}[1]{\label{sec:#1}}
\newcommand{\refEq}[1]{Ecuaci�n~\eqref{eq:#1}}
\newcommand{\refFig}[1]{Fig.\ref{fig:#1}}
\newcommand{\refTab}[1]{tabla~\ref{tab:#1}}
\newcommand{\RefTab}[1]{Tabla~\ref{tab:#1}}
\newcommand{\refSec}[1]{secci�n~\ref{sec:#1}}
\newcommand{\RefSec}[1]{Secci�n~\ref{sec:#1}}
\newcommand{\refApe}[1]{ap�ndice~\ref{sec:#1}}
\newcommand{\RefApe}[1]{Ap�ndice~\ref{sec:#1}}
\newcommand{\angl}[1]{{\em #1}}
\newcommand{\cod}[1]{{\tt #1}}

%% I have got rid of the .eps file
%% extension in the filename, in order to allow PDFLATEX to
%% import the PDF's, which I have also created and included
%% in the zip file.
%% I normally use this macro for this:
%% \newFig{filename}{Caption}
%% I use the <filename> as the label... it's easier to
%% remember for me...





\ifpdf
\newcommand{\newFig}[2] {
\begin{figure}[ht]
\centering
\includegraphics [width=0.40\textwidth] {Figs/pdf/#1}
\caption {#2}
\labelFig{#1}
\end{figure}
}

\newcommand{\newFigWidth}[3] {
\begin{figure}[ht]
\centering
\includegraphics [width=#1\textwidth] {Figs/pdf/#2}
\caption {#3}
\labelFig{#2}
\end{figure}
}


\newcommand{\newFigTwoXTree}[8] {
\begin{figure}[ht!]
  \centering
   %%----1x1----
  \subfloat[]{
        \labelFig{#3}         %% Etiqueta para la primera subfigura
        \includegraphics[width=0.21\textwidth]{Figs/pdf/#3}}
   %%----1x2----
  \subfloat[]{
        \labelFig{#4}         %% Etiqueta para la segunda subfigura
        \includegraphics[width=0.21\textwidth]{Figs/pdf/#4}}\\[20pt]
   %%----2x1----
  \subfloat[]{
        \labelFig{#5}         %% Etiqueta para la primera subfigura
        \includegraphics[width=0.21\textwidth]{Figs/pdf/#5}}
   %%----2x2----
  \subfloat[]{
        \labelFig{#6}         %% Etiqueta para la segunda subfigura
        \includegraphics[width=0.21\textwidth]{Figs/pdf/#6}}\\[20pt]
   %%----3x1----
  \subfloat[]{
        \labelFig{#7}         %% Etiqueta para la primera subfigura
        \includegraphics[width=0.21\textwidth]{Figs/pdf/#7}}
   %%----3x2----
  \subfloat[]{
        \labelFig{#8}         %% Etiqueta para la segunda subfigura
        \includegraphics[width=0.21\textwidth]{Figs/pdf/#8}}
  \labelFig{#1}                %% Etiqueta para la figura entera
  \caption{#2}
\end{figure}
}







\else
\newcommand{\newFig}[2] {
\begin{figure}[ht]
\centering
\includegraphics [width=0.40\textwidth] {Figs/eps/#1}
\caption {#2}
\labelFig{#1}
\end{figure}
}

\newcommand{\newFigWidth}[3] {
\begin{figure}[ht]
\centering
\includegraphics [width=#1\textwidth] {Figs/eps/#2}
\caption {#3}
\labelFig{#2}
\end{figure}
}
\fi






% Some very useful LaTeX packages include:
% (uncomment the ones you want to load)


% *** MISC UTILITY PACKAGES ***
%
%\usepackage{ifpdf}
% Heiko Oberdiek's ifpdf.sty is very useful if you need conditional
% compilation based on whether the output is pdf or dvi.
% usage:
% \ifpdf
%   % pdf code
% \else
%   % dvi code
% \fi
% The latest version of ifpdf.sty can be obtained from:
% http://www.ctan.org/tex-archive/macros/latex/contrib/oberdiek/
% Also, note that IEEEtran.cls V1.7 and later provides a builtin
% \ifCLASSINFOpdf conditional that works the same way.
% When switching from latex to pdflatex and vice-versa, the compiler may
% have to be run twice to clear warning/error messages.






% *** CITATION PACKAGES ***
%
%\usepackage{cite}
% cite.sty was written by Donald Arseneau
% V1.6 and later of IEEEtran pre-defines the format of the cite.sty package
% \cite{} output to follow that of IEEE. Loading the cite package will
% result in citation numbers being automatically sorted and properly
% "compressed/ranged". e.g., [1], [9], [2], [7], [5], [6] without using
% cite.sty will become [1], [2], [5]--[7], [9] using cite.sty. cite.sty's
% \cite will automatically add leading space, if needed. Use cite.sty's
% noadjust option (cite.sty V3.8 and later) if you want to turn this off.
% cite.sty is already installed on most LaTeX systems. Be sure and use
% version 4.0 (2003-05-27) and later if using hyperref.sty. cite.sty does
% not currently provide for hyperlinked citations.
% The latest version can be obtained at:
% http://www.ctan.org/tex-archive/macros/latex/contrib/cite/
% The documentation is contained in the cite.sty file itself.






% *** GRAPHICS RELATED PACKAGES ***
%
\ifCLASSINFOpdf
  % \usepackage[pdftex]{graphicx}
  % declare the path(s) where your graphic files are
  % \graphicspath{{../pdf/}{../jpeg/}}
  % and their extensions so you won't have to specify these with
  % every instance of \includegraphics
  % \DeclareGraphicsExtensions{.pdf,.jpeg,.png}
\else
  % or other class option (dvipsone, dvipdf, if not using dvips). graphicx
  % will default to the driver specified in the system graphics.cfg if no
  % driver is specified.
  % \usepackage[dvips]{graphicx}
  % declare the path(s) where your graphic files are
  % \graphicspath{{../eps/}}
  % and their extensions so you won't have to specify these with
  % every instance of \includegraphics
  % \DeclareGraphicsExtensions{.eps}
\fi
% graphicx was written by David Carlisle and Sebastian Rahtz. It is
% required if you want graphics, photos, etc. graphicx.sty is already
% installed on most LaTeX systems. The latest version and documentation can
% be obtained at: 
% http://www.ctan.org/tex-archive/macros/latex/required/graphics/
% Another good source of documentation is "Using Imported Graphics in
% LaTeX2e" by Keith Reckdahl which can be found as epslatex.ps or
% epslatex.pdf at: http://www.ctan.org/tex-archive/info/
%
% latex, and pdflatex in dvi mode, support graphics in encapsulated
% postscript (.eps) format. pdflatex in pdf mode supports graphics
% in .pdf, .jpeg, .png and .mps (metapost) formats. Users should ensure
% that all non-photo figures use a vector format (.eps, .pdf, .mps) and
% not a bitmapped formats (.jpeg, .png). IEEE frowns on bitmapped formats
% which can result in "jaggedy"/blurry rendering of lines and letters as
% well as large increases in file sizes.
%
% You can find documentation about the pdfTeX application at:
% http://www.tug.org/applications/pdftex





% *** MATH PACKAGES ***
%
%\usepackage[cmex10]{amsmath}
% A popular package from the American Mathematical Society that provides
% many useful and powerful commands for dealing with mathematics. If using
% it, be sure to load this package with the cmex10 option to ensure that
% only type 1 fonts will utilized at all point sizes. Without this option,
% it is possible that some math symbols, particularly those within
% footnotes, will be rendered in bitmap form which will result in a
% document that can not be IEEE Xplore compliant!
%
% Also, note that the amsmath package sets \interdisplaylinepenalty to 10000
% thus preventing page breaks from occurring within multiline equations. Use:
%\interdisplaylinepenalty=2500
% after loading amsmath to restore such page breaks as IEEEtran.cls normally
% does. amsmath.sty is already installed on most LaTeX systems. The latest
% version and documentation can be obtained at:
% http://www.ctan.org/tex-archive/macros/latex/required/amslatex/math/





% *** SPECIALIZED LIST PACKAGES ***
%
%\usepackage{algorithmic}
% algorithmic.sty was written by Peter Williams and Rogerio Brito.
% This package provides an algorithmic environment fo describing algorithms.
% You can use the algorithmic environment in-text or within a figure
% environment to provide for a floating algorithm. Do NOT use the algorithm
% floating environment provided by algorithm.sty (by the same authors) or
% algorithm2e.sty (by Christophe Fiorio) as IEEE does not use dedicated
% algorithm float types and packages that provide these will not provide
% correct IEEE style captions. The latest version and documentation of
% algorithmic.sty can be obtained at:
% http://www.ctan.org/tex-archive/macros/latex/contrib/algorithms/
% There is also a support site at:
% http://algorithms.berlios.de/index.html
% Also of interest may be the (relatively newer and more customizable)
% algorithmicx.sty package by Szasz Janos:
% http://www.ctan.org/tex-archive/macros/latex/contrib/algorithmicx/




% *** ALIGNMENT PACKAGES ***
%
%\usepackage{array}
% Frank Mittelbach's and David Carlisle's array.sty patches and improves
% the standard LaTeX2e array and tabular environments to provide better
% appearance and additional user controls. As the default LaTeX2e table
% generation code is lacking to the point of almost being broken with
% respect to the quality of the end results, all users are strongly
% advised to use an enhanced (at the very least that provided by array.sty)
% set of table tools. array.sty is already installed on most systems. The
% latest version and documentation can be obtained at:
% http://www.ctan.org/tex-archive/macros/latex/required/tools/


%\usepackage{mdwmath}
%\usepackage{mdwtab}
% Also highly recommended is Mark Wooding's extremely powerful MDW tools,
% especially mdwmath.sty and mdwtab.sty which are used to format equations
% and tables, respectively. The MDWtools set is already installed on most
% LaTeX systems. The lastest version and documentation is available at:
% http://www.ctan.org/tex-archive/macros/latex/contrib/mdwtools/


% IEEEtran contains the IEEEeqnarray family of commands that can be used to
% generate multiline equations as well as matrices, tables, etc., of high
% quality.


%\usepackage{eqparbox}
% Also of notable interest is Scott Pakin's eqparbox package for creating
% (automatically sized) equal width boxes - aka "natural width parboxes".
% Available at:
% http://www.ctan.org/tex-archive/macros/latex/contrib/eqparbox/





% *** SUBFIGURE PACKAGES ***
%\usepackage[tight,footnotesize]{subfigure}
% subfigure.sty was written by Steven Douglas Cochran. This package makes it
% easy to put subfigures in your figures. e.g., "Figure 1a and 1b". For IEEE
% work, it is a good idea to load it with the tight package option to reduce
% the amount of white space around the subfigures. subfigure.sty is already
% installed on most LaTeX systems. The latest version and documentation can
% be obtained at:
% http://www.ctan.org/tex-archive/obsolete/macros/latex/contrib/subfigure/
% subfigure.sty has been superceeded by subfig.sty.



%\usepackage[caption=false]{caption}
%\usepackage[font=footnotesize]{subfig}
% subfig.sty, also written by Steven Douglas Cochran, is the modern
% replacement for subfigure.sty. However, subfig.sty requires and
% automatically loads Axel Sommerfeldt's caption.sty which will override
% IEEEtran.cls handling of captions and this will result in nonIEEE style
% figure/table captions. To prevent this problem, be sure and preload
% caption.sty with its "caption=false" package option. This is will preserve
% IEEEtran.cls handing of captions. Version 1.3 (2005/06/28) and later 
% (recommended due to many improvements over 1.2) of subfig.sty supports
% the caption=false option directly:
%\usepackage[caption=false,font=footnotesize]{subfig}
%
% The latest version and documentation can be obtained at:
% http://www.ctan.org/tex-archive/macros/latex/contrib/subfig/
% The latest version and documentation of caption.sty can be obtained at:
% http://www.ctan.org/tex-archive/macros/latex/contrib/caption/




% *** FLOAT PACKAGES ***
%
%\usepackage{fixltx2e}
% fixltx2e, the successor to the earlier fix2col.sty, was written by
% Frank Mittelbach and David Carlisle. This package corrects a few problems
% in the LaTeX2e kernel, the most notable of which is that in current
% LaTeX2e releases, the ordering of single and double column floats is not
% guaranteed to be preserved. Thus, an unpatched LaTeX2e can allow a
% single column figure to be placed prior to an earlier double column
% figure. The latest version and documentation can be found at:
% http://www.ctan.org/tex-archive/macros/latex/base/



%\usepackage{stfloats}
% stfloats.sty was written by Sigitas Tolusis. This package gives LaTeX2e
% the ability to do double column floats at the bottom of the page as well
% as the top. (e.g., "\begin{figure*}[!b]" is not normally possible in
% LaTeX2e). It also provides a command:
%\fnbelowfloat
% to enable the placement of footnotes below bottom floats (the standard
% LaTeX2e kernel puts them above bottom floats). This is an invasive package
% which rewrites many portions of the LaTeX2e float routines. It may not work
% with other packages that modify the LaTeX2e float routines. The latest
% version and documentation can be obtained at:
% http://www.ctan.org/tex-archive/macros/latex/contrib/sttools/
% Documentation is contained in the stfloats.sty comments as well as in the
% presfull.pdf file. Do not use the stfloats baselinefloat ability as IEEE
% does not allow \baselineskip to stretch. Authors submitting work to the
% IEEE should note that IEEE rarely uses double column equations and
% that authors should try to avoid such use. Do not be tempted to use the
% cuted.sty or midfloat.sty packages (also by Sigitas Tolusis) as IEEE does
% not format its papers in such ways.


%\ifCLASSOPTIONcaptionsoff
%  \usepackage[nomarkers]{endfloat}
% \let\MYoriglatexcaption\caption
% \renewcommand{\caption}[2][\relax]{\MYoriglatexcaption[#2]{#2}}
%\fi
% endfloat.sty was written by James Darrell McCauley and Jeff Goldberg.
% This package may be useful when used in conjunction with IEEEtran.cls'
% captionsoff option. Some IEEE journals/societies require that submissions
% have lists of figures/tables at the end of the paper and that
% figures/tables without any captions are placed on a page by themselves at
% the end of the document. If needed, the draftcls IEEEtran class option or
% \CLASSINPUTbaselinestretch interface can be used to increase the line
% spacing as well. Be sure and use the nomarkers option of endfloat to
% prevent endfloat from "marking" where the figures would have been placed
% in the text. The two hack lines of code above are a slight modification of
% that suggested by in the endfloat docs (section 8.3.1) to ensure that
% the full captions always appear in the list of figures/tables - even if
% the user used the short optional argument of \caption[]{}.
% IEEE papers do not typically make use of \caption[]'s optional argument,
% so this should not be an issue. A similar trick can be used to disable
% captions of packages such as subfig.sty that lack options to turn off
% the subcaptions:
% For subfig.sty:
% \let\MYorigsubfloat\subfloat
% \renewcommand{\subfloat}[2][\relax]{\MYorigsubfloat[]{#2}}
% For subfigure.sty:
% \let\MYorigsubfigure\subfigure
% \renewcommand{\subfigure}[2][\relax]{\MYorigsubfigure[]{#2}}
% However, the above trick will not work if both optional arguments of
% the \subfloat/subfig command are used. Furthermore, there needs to be a
% description of each subfigure *somewhere* and endfloat does not add
% subfigure captions to its list of figures. Thus, the best approach is to
% avoid the use of subfigure captions (many IEEE journals avoid them anyway)
% and instead reference/explain all the subfigures within the main caption.
% The latest version of endfloat.sty and its documentation can obtained at:
% http://www.ctan.org/tex-archive/macros/latex/contrib/endfloat/
%
% The IEEEtran \ifCLASSOPTIONcaptionsoff conditional can also be used
% later in the document, say, to conditionally put the References on a 
% page by themselves.





% *** PDF, URL AND HYPERLINK PACKAGES ***
%
%\usepackage{url}
% url.sty was written by Donald Arseneau. It provides better support for
% handling and breaking URLs. url.sty is already installed on most LaTeX
% systems. The latest version can be obtained at:
% http://www.ctan.org/tex-archive/macros/latex/contrib/misc/
% Read the url.sty source comments for usage information. Basically,
% \url{my_url_here}.





% *** Do not adjust lengths that control margins, column widths, etc. ***
% *** Do not use packages that alter fonts (such as pslatex).         ***
% There should be no need to do such things with IEEEtran.cls V1.6 and later.
% (Unless specifically asked to do so by the journal or conference you plan
% to submit to, of course. )




\begin{document}
%
% paper title
% can use linebreaks \\ within to get better formatting as desired
\title{Paralelizaci�n en GPU del algoritmo Shell Sort}
%
%
% author names and IEEE memberships
% note positions of commas and nonbreaking spaces ( ~ ) LaTeX will not break
% a structure at a ~ so this keeps an author's name from being broken across
% two lines.
% use \thanks{} to gain access to the first footnote area
% a separate \thanks must be used for each paragraph as LaTeX2e's \thanks
% was not built to handle multiple paragraphs
%

\author{Francisco Huertas Ferrer}

% note the % following the last \IEEEmembership and also \thanks - 
% these prevent an unwanted space from occurring between the last author name
% and the end of the author line. i.e., if you had this:
% 
% \author{....lastname \thanks{...} \thanks{...} }
%                     ^------------^------------^----Do not want these spaces!
%
% a space would be appended to the last name and could cause every name on that
% line to be shifted left slightly. This is one of those "LaTeX things". For
% instance, "\textbf{A} \textbf{B}" will typeset as "A B" not "AB". To get
% "AB" then you have to do: "\textbf{A}\textbf{B}"
% \thanks is no different in this regard, so shield the last } of each \thanks
% that ends a line with a % and do not let a space in before the next \thanks.
% Spaces after \IEEEmembership other than the last one are OK (and needed) as
% you are supposed to have spaces between the names. For what it is worth,
% this is a minor point as most people would not even notice if the said evil
% space somehow managed to creep in.



% The paper headers
% Cambiados los headers para que pueda seleccionarse en espa�ol
\renewcommand{\leftmark}{Paralelizaci�n en GPU del algoritmo Shell Sort}
\renewcommand{\rightmark}{Francisco Huertas Ferrer}
% \markboth{Journal of \LaTeX\ Class Files,~Vol.~6, No.~1, January~2007}%
% {Shell \MakeLowercase{\textit{et al.}}: Bare Demo of IEEEtran.cls for Journals}
% The only time the second header will appear is for the odd numbered pages
% after the title page when using the twoside option.
% 
% *** Note that you probably will NOT want to include the author's ***
% *** name in the headers of peer review papers.                   ***
% You can use \ifCLASSOPTIONpeerreview for conditional compilation here if
% you desire.




% If you want to put a publisher's ID mark on the page you can do it like
% this:
%\IEEEpubid{0000--0000/00\$00.00~\copyright~2007 IEEE}
% Remember, if you use this you must call \IEEEpubidadjcol in the second
% column for its text to clear the IEEEpubid mark.



% use for special paper notices
%\IEEEspecialpapernotice{(Invited Paper)}




% make the title area
\maketitle


\begin{abstract}

El desarrollo de software est� sufriendo un cambio de concepto en favor de sistemas concurrentes.
Este cambio de concepto se debe en gran medida a necesidad de mejorar las prestaciones actuales, ya
sea para aumentar la capacidad de computo, mejorar la productividad de las aplicaciones o disponer
de medidas de respaldo de sistemas. Estos sistemas concurrentes obligan a un cambio
de concepto a la hora de desarrollar nuevo software y programar algoritmos.

El sistema de concurrencia a nivel de n�cleo de computo que ofrecen las arquitecturas de
c�lculo paralelo presentes en las unidades de procesamiento gr�fico (Graphical Process Unit,
GPU), posee una gran potencia de computo paralelo debido al inherente grado de paralelismo que
poseen el procesamiento gr�fico. 

Con la aparici�n de lenguajes de programaci�n como OpenCL y CUDA que permite delegar trabajo a la
GPU aparece la posibilidades de utilizarlas para realizar c�mputo paralelo de datos. Este tipo de
programaci�n plantea un nuevo reto para programar de nuevo algoritmos y optimizarlos al paralelismo
ofrecido por la arquitectura de estos dispositivos. Este reto conlleva tambi�n un cambio de
mentalidad la hora de realizar las implementaciones.

En este trabajo se ha buscado la paralelizaci�n del algoritmo de ordenaci�n Shellsort y su
implementaci�n del mismo bajo OpenCL. 

\end{abstract}

% IEEEtran.cls defaults to using nonbold math in the Abstract.
% This preserves the distinction between vectors and scalars. However,
% if the journal you are submitting to favors bold math in the abstract,
% then you can use LaTeX's standard command \boldmath at the very start
% of the abstract to achieve this. Many IEEE journals frown on math
% in the abstract anyway.

% Note that keywords are not normally used for peerreview papers.
\begin{IEEEkeywords}
OpenCL, CUDA, Shellsort, paralelo
\end{IEEEkeywords}






% For peer review papers, you can put extra information on the cover
% page as needed:
% \ifCLASSOPTIONpeerreview
% \begin{center} \bfseries EDICS Category: 3-BBND \end{center}
% \fi
%
% For peerreview papers, this IEEEtran command inserts a page break and
% creates the second title. It will be ignored for other modes.
\IEEEpeerreviewmaketitle

\section{Introducci�n}

\subsection{Potencia en unidades de procesamiento gr�fico, procesamiento paralelo}

Durante la �ltima d�cada, las \acp{GPU} han sufrido una gran evoluci�n ofreciendo una capacidad de
procesamiento muy superior a las \acp{CPU}. Esta superioridad esta basada en la peculiaridad de las
operaciones de procesamiento de im�genes, que se consiste, en muchos
casos, en realizar la misma operaci�n para cada uno de los puntos de una imagen. 

\newFig{arquitectura}{Diferencias arquitect�nicas entre una \ac{CPU} y una \ac{GPU}}

Aprovechando esta caracter�sticas, las \acp{GPU} basan su capacidad de computo en la utilizaci�n de
gran cantidad de n�cleos especializados que trabajan en paralelo. La arquitectura de estos n�cleos
es mucho m�s sencilla que las que tiene una \acp{CPU} como muestra la \refFig{arquitectura}. Esta
arquitectura, por tanto, no es apta para programaci�n general.

\subsection{Unidades de procesamiento gr�fico y programaci�n general}

La mejora de prestaciones de las \acp{GPU} ha aumentando el inter�s para el uso de
estos dispositivos en otro tipo de programas. Este hecho ha producido la aparici�n de
nuevos lenguajes de programaci�n como CUDA y OpenCL que faciliten el uso de los dispositivos. 

Actualmente est�n apareciendo implementaciones de algoritmos adaptados para su ejecuci�n paralela
que aprovechen la potencia de estos dispositivos gr�ficos. Estos algoritmos se caracterizan por
ser altamente paralelizables pudiendo aprovechar al m�ximo la potencia de calcula de las \acp{GPU}.




\section{Paralelizaci�n de Shellsort, An�lisis}

Los algoritmos con mayor nivel de paralelizaci�n ya se  encuentra n implementado s,  po r lo que
encontrar un algoritmo que pueda implementarse de manera eficiente de forma paralela
resulta complejo.

Al buscar un algoritmo para su paralelizaci�n se ha tenido en cuenta varias opciones, estudiando,
entre los algoritmos de busqueda y ordenaci�n, aquellos que no poseen todav�a una implementaciones
para OpenCL o CUDA. Entre estos algoritmos se ha elegido Shellsort debido al aparente caracter
paralelo de sus operaciones de comparaci�n entre elementos deslocalizados en su estructura de
datos. 

\subsection{Shellsort}

El algoritmo de ordenamiento Shellsort recibe su nombre en honor a su inventor Donald Shell. Su
implementaci�n original, requiere $O(n^2)$ comparaciones e intercambios en el peor caso. El
algoritmo ha sido mejorado en diferentes implementaciones obteniendo rendimientos de $O(n log^2
(n))$ y $O(n^{3/4})$ en el peor caso. 

El Shell sort es una generalizaci�n del ordenamiento por inserci�n, teniendo en cuenta dos
observaciones:
\begin{enumerate}
 \item El ordenamiento por inserci�n es eficiente si la entrada est� ``casi ordenada''.
 \item El ordenamiento por inserci�n es ineficiente, en general, porque mueve los valores s�lo una
posici�n cada vez.
\end{enumerate}

El algoritmo Shellsort mejora el ordenamiento por inserci�n comparando elementos separados por un
espacio de varias posiciones. Esto permite que un elemento haga ``pasos m�s grandes'' hacia su
posici�n esperada. Los pasos m�ltiples sobre los datos se hacen con tama�os de espacio cada vez m�s
peque�os. El �ltimo paso del Shellsort es un simple ordenamiento por inserci�n, pero para entonces,
ya est� garantizado que los datos del vector est�n casi ordenados.

El rendimiento del algoritmo depende directamente de la separaci�n entre los elementos comparados.
Los mejores resultados hasta la fecha se obtienen usando los incrementos de Sedgewick con un coste
de $O(n log^2 (n))$. 

\subsection{Requisitos de paralelizaci�n}

La arquitectura particular de los dispositivos gr�ficos hacen que no todos los algoritmos sean
elegibles para su paralelizaci�n. Estos requisitos indican cuales son los algoritmos que mejor se
adaptan a la paralelizaci�n. 

\subsubsection{Operaciones}

Para que un algoritmo pueda alcanzar un buen rendimiento paralelo es
necesario que existan operaciones que se ejecuten muchas veces con diferentes datos y que la
dependencia entre la misma operaci�n con distintos datos no sea muy fuerte. Un ejemplo de esto es
incrementar en uno el valor de un grupo de n�meros. Esta operaci�n no posee dependencias entre su
ejecuci�n con distintos datos y es repetida tantas veces como numero de datos existan.  

\subsubsection{Conjunto de datos} 

La paralelizaci�n de un algoritmo se basa en la repetici�n de una
varias operaciones sobre un conjunto de datos. Para que sea interesante ejecutar el algoritmo en
un dispositivo gr�fico, debe ejecutarse sobre un grupo de datos lo suficientemente grande. 

\subsubsection{Representaci�n de los datos} 
Los dispositivos gr�ficos est�n optimizados para tratar con im�genes y v�deos. Si los datos del
algoritmo est�n distribuidos en estructuras que se asemejen a las utilizadas por im�genes y v�deos,
el acceso a los mismos resultar� mucho m�s sencillo. Estas estructuras son principalmente matrices
de varias dimensiones. 


Estos requisitos marcan los resultados positivos que se pueden obtener al paralelizar un algoritmo
concreto

\subsection{An�lisis y valoraci�n}

El an�lisis de los requisitos en el algoritmo Shellsort se valora cada uno de los requisitos de
paralelizaci�n:

\subsubsection{Operaciones}

El algoritmo se basa en el algoritmo de inserci�n, este no se caracteriza por ser un algoritmo
paralelizable, sin embargo Shellsort se caracteriza por realizar comparaciones entre subgrupos de
elementos. Los subconjunto de elementos son m�s numerosos y con menos elementos al principio del
algoritmo. En la �ltima etapa del algoritmo, existe un solo subconjunto con todos los elementos. 

Los primeros pasos de Shellsort, al existir subconjuntos independientes de datos a comparar, la
paralelizaci�n es sencilla, sin embargo en los �ltimos pasos. Al existir pocos subconjuntos es
necesario buscar una forma de paralelizar de forma eficiente el algoritmo de inserci�n para listas
de elementos ``casi ordenadas''. 

\subsubsection{Conjunto de datos}

Al ser un algoritmo de ordenaci�n, el conjunto de datos es lo suficientemente grande para realizar
una implementaci�n paralela del algoritmo. 

\subsubsection{Representaci�n de los datos}

La representaci�n de los datos en una primera instancia puede favorecer una implementaci�n
paralela. Los datos se ordenan en una lista y los subconjuntos de datos se encuentran a distancias
fijas. 

Sin embargo hay caracter�sticas que van a afectar negativamente a este aspecto. El primero es que
el tama�o de los subgrupos va a ser muy diferente en los diferentes pasos del algoritmo. Esto va a
producir que la implementaci�n tenga que adaptarse a este cambio. 

Otro aspecto que puede afectar negativamente al algoritmo es que los elementos de cada cada subgrupo
tiene sus elementos muy deslocalizados. 


\subsection{Conclusi�n del an�lisis}

Shellsort, pese a poseer caracter�sticas que aparentemente lo hacen paralelizable, tiene
aspectos mermar�n la implementaci�n paralela que se realice y, por tanto, la implementaci�n
no se beneficiara de la paralelizaci�n tanto como otros algoritmos. Esta puede ser la causa por la
cual no exista a�n una implementaci�n de Shellsort para OpenCL o CUDA. 










\section{Paralelizaci�n de Shellsort, Implementaci�n }

La implementaci�n paralela de Shellsort se ha realizado por fases. Cada una de las fases se ha
centrado en un aspecto de la paralelizaci�n. Tambi�n se ha hecho una implementaci�n lineal para
\ac{CPU} y otra para \ac{GPU} para comprar los resultados. 

El c�digo de la \ac{GPU} se ha realizado en OpenCL. Se ha utilizado C para escribir el c�digo que
prepara el contexto OpenCL y lanza los hilos que se ejecutan en la \ac{GPU}. La compilaci�n se ha
hecho en Linux. Los incrementos utilizados en el algoritmo son los incrementos de Sedgewick.

Por �ltimo, no todas las implementaciones ordenan por completo el vector de elementos.

\subsection{Implementaci�n lineal base, Shellsort 0}

Esta implementaci�n se ha utilizado como base para las siguientes implementaciones. Aunque se
ejecute en la \ac{GPU} el c�digo se ejecuta de manera lineal.

El c�digo que se ejecuta en la \ac{CPU} es el encargado para prepara el contexto para la ejecuci�n
del c�digo OpenCL. Tambi�n se encarga de calcular los distintos incrementos. 

El c�digo para que se ejecuta en la \ac{GPU} es el encargado de ordenar el subconjunto de elementos
del vector que recibe. Los par�metros necesarios para realizar la llamada son: 

\begin{itemize}
 \item El propio vector de elementos. En dicho vector est�n todos los elementos del vector ({\em
vector})
 \item El tama�o del vector ({\em size}) 
 \item El primer elemento que hay que ordenar ({\em offset}) 
 \item La separaci�n entre los elementos que hay que ordenar ({\em incremento}) 
\end{itemize}

La ejecuci�n del c�digo OpenCL se realiza de manera secuencial para cada incremento y offset y se
realiza en una dimensi�n y un �nico hilo por dimensi�n. 

Esta implementaci�n ordena los elementos del vector por completo

\subsection{Paralelizaci�n de los incrementos, Shellsort A}

Este nivel de paralelizaci�n es el m�s f�cil de realizar. A la hora de lanzar los hilos en la
\ac{GPU} se lanzan simult�neamente un hilo por cada subconjunto de elementos. Por cada fase del
algoritmo se lanzan n hilos donde n es el valor del incremento de Sedgewick para esa fase, en
concreto las llamadas se realizan en una �nica dimensi�n de n elementos y no existen comunicaci�n
entre los hilos y por lo tanto cada hilo se ejecuta en un grupo distinto. Las fases se siguen
ejecutando de manera secuencial. 

En la implementaci�n OpenCL se sustituye el parametro offset que indicaba el subconjunto que se
ordenaba por el id global del hilo en la �nica dimensi�n existente.

Esta paralelizaci�n, para las primeras fases del algoritmo obtiene muy buenos resultados, sin
embargo, en las ultimas fases el algoritmo no obtiene grandes resultados ya que el numero de hilos
lanzados es muy peque�o. 

Tomando los incrementos de Sedgewick (1, 5, 19, 41, 109, ...), para una lista de 100 elementos
cada fase posee las siguientes caracter�sticas: 
\begin{itemize}
 \item 1�: 41 hilos, 2-3 elementos ordenados por hilo. 
 \item 2�: 19 hilos, 5-6 elementos ordenados por hilo. 
 \item 3�: 5 hilos, 20 elementos ordenados por hilo
 \item 4�: 1 hilos, 100 elementos ordenados por hilo
\end{itemize}

Esta implementaci�n ordena los elementos del vector por completo. 

\subsection{Paralelizaci�n del algoritmo de inserci�n, Shellsort B}

Con la base del apartado anterior se ha buscado una forma de mejorar el rendimiento cuando el
n�mero de subconjunto es peque�o. Para realizar esto se han lanzado un n�mero variable de hilos por
subconjunto que son los encargados de ordenar un peque�o grupo de elementos pertenecientes a dicho
subconjunto. 

La implementaci�n lanza los hilos en dos dimensiones: 
\begin{itemize}
 \item La primera representa el subconjunto que ordena cada hilo. 
 \item La segunda representa que parte del subconjunto ordena. 
\end{itemize}

En esta implementaci�n no es necesario clasificar los hilos por grupos, sin embargo, por claridad y
se han agrupado los hilos en grupos que ordenan el mismo subconjunto de elementos del vector.
De igual manera es aconsejable determinar un n�mero m�ximo de hilos por subconjunto. este valor
depende del n�mero de elementos del subconjunto y nunca puede superar el tama�o m�ximo por
dimensi�n por grupo de trabajo. 

Tomando de nuevo los incrementos de Sedgewick (1, 5, 19, 41, 109, ...), para una lista de 100
elementos y estableciendo como m�ximo de hilos por grupo una cuarta parte de los elementos por
grupo, las fases del algoritmo tienen las siguientes caracter�sticas: 

\begin{itemize}
 \item 1�: 41x1 hilos (41 en total), 2-3 elementos ordenados por hilo. 
 \item 2�: 19x1 hilos (19 en total), 4-5 elementos ordenados por hilo.
 \item 3�: 5x5 hilos (25 en total),4 elementos ordenados por hilo.
 \item 4�: 1x25 hilos (25 hilos en total), 4 elementos ordenados por hilo.
\end{itemize}

Este algoritmo mejora notablemente el rendimiento con respecto a la versi�n anterior. Sin embargo
esta implementaci�n no ordena el por completo el vector de elementos, ya que no existe comunicaci�n
entre los hilos que ordenan el mismo subconjunto de elementos. 


\subsection{Variante de Shellsort, Shellsort C }

Esta versi�n �nicamente cambia el orden en que se ejecuta el algoritmo de inserci�n. En el apartado
anterior se ejecuta del elemento menor al mayor. En este se ejecuta del mayor al menor. Al obtener
resultados similares esta versi�n se ha descontinuado. 

\subsection{Correcci�n del algoritmo de inserci�n, Shellsort D}

Esta correcci�n busca que el algoritmo planteado en Shellsort B ordene los elementos

La paralizaci�n Shellsort B deja los elementos ordenados por hilo. Para ordenar todos los elementos
de un subconjunto se comparan el elemento mayor y menor de los hilos consecutivos. En caso de que
no est�n ordenados se intercambian los valores y comienza de nuevo a ordenarse cada hilo
internamente. 

Aparentemente esta forma de ordenar no es muy eficiente, sin embargo, se ha optado por esta
versi�n porque los elementos de cada fase se encuentran m�s cerca de su posici�n real y, por lo
tanto no, habr� hacer hacer muchas repeticiones. 

Se ha planteado realizar una modificaci�n de realizando m�s de un cambio de elementos entre hilos
por pasada. Al final se descart� por no conseguir resultados �ptimos.  

Otra mejora que se plante� en este apartado es desactivar hilos que, aparentemente no necesiten
volver a ordenar, sin embargo, al realizar un cambio en cualquiera de los elementos, cualquier hilo
es sensible de necesitar reordenar sus elementos. 

Por �ltimo a�adir que se han incluido optimizaciones a la hora de traer los elementos de memoria.
Estas mejoras intentan hacer el m�nimo n�mero de lecturas y escrituras en memoria global del
dispositivo. 

\subsection{Uso de memoria local, Shellsort E y F}

Un de los problemas que suele tener las implementaciones de algoritmos para OpenCL o CUDA es la
gran cantidad de tiempo que se pierde accediendo a memoria general. Como se podr� ver en la secci�n
siguiente, el numero de accesos a memoria en la implementaci�n Shellsort D puede llegar a ser muy
alta. Para evitar esto se ha intentado mover el vector de memoria general a memoria privada. 

El primer problema que se ha tenido que solventar a sido que en las �ltimas fases, que generalmente
son las m�s lentas, tienen una cantidad tan grande de datos que es imposible almacenar en memoria
toda el vector. Para solventar este problema se ha hecho que cada hilo trabaje en un grupo
distinto. La pega de esto es que ahora todos los hilos tienen que esperarse para comprobar si ha
habido cambios en su subconjunto de datos. Cuando un hilo ha terminado de ordenar sus datos realiza
la copia en memoria general del mayor y menor de sus datos. Una vez hecho esto se comprueban los
datos entre hilos para saber si hay que hacer intercambio. 

Estas implementaciones, sin embargo, no ha mejorado nada el rendimiento con respecto a las
versiones anteriores. Los principales problemas que se han encontrado es que al variar tanto las
dimensiones tan variables con las que se trabaja. 
















\section{Estudio}

Las pruebas que se han hecho para comprobar las distintas implementaciones se han basado en pruebas
emp�ricas obtenidas al ejecutar el c�digo en diferentes equipos. 

  Para comprobar

\subsection{Hardware utilizado}

Todas las implementaciones se han probado en al menos un equipo. Adicionalmente, las mejores
implementaciones se han probado en dos equipos m�s. Las especificaciones de los equipos se
encuentran en el ap�ndice A. 

Una de las principales pegas de las pruebas realizadas es la imposibilidad de probarlo en
dispositivos gr�ficos de alto rendimiento. Las pruebas se han hecho principalmente en el Equipo A.
El dispositivo gr�fico de este equipo solo dispone de 16 n�cleos CUDA cuando los dispositivos
actuales de gama alta disponen de miles de n�cleos. Por otro lado el procesador es, con respecto a
la tarjeta gr�fica, m�s avanzado. 

\subsection{Varios tiempos un solo equipo }

\begin{table}
\begin{center}
\begin{tabular}{|l | c | c| c | }
\hline
Elementos x& & & \\ 
repeticiones & Lineal & Imp-0 & Imp-A \\ 
\hline
100 x 1       & 0.00  & 0.01  & 0.00  \\ 
1000 x 1      & 0.00  & 0.63  & 0.15  \\ 
10 x 100      & 0.00  & 0.14  & 0.04  \\ 
100 x 100     & 0.00  & 18.25 & 0.24  \\ 
1000 x 100    & 0.02  & ----- & 14.56 \\ 
10000 x 100   & 0.30  & ----- & ----- \\ 
100000 x 100  & 3.76  & ----- & ----- \\ 
400000 x 100  & 19.08 & ----- & ----- \\ 
1000000 x 100 & 55.74 & ----- & ----- \\ 
\hline
\hline
Elementos x& & & \\ 
repeticiones  & Imp-B & Imp-D & Imp-E\\
\hline
100 x 1       & 0.00   & 0.00   & 0.00 \\ 
1000 x 1      & 0.00   & 0.00   & 0.00\\ 
10 x 100      & 0.04   & 0.07   & 0.04\\ 
100 x 100     & 0.08   & 0.14   & 0.08\\ 
1000 x 100    & 0.10   & 0.34   & 0.17\\ 
10000 x 100   & 0.73   & 1.31   & 1.57\\ 
100000 x 100  & 10.78  & 15.08  & 21.32\\ 
400000 x 100  & 58.49  & 69.07  & 95.02\\ 
1000000 x 100 & 183.98 & 180.93 & -----\\ 
\hline
\end{tabular}
\caption{Tiempos de ordenaci�n en el equipo A}
\labelTab{TiemposGruposA}
\end{center}
\end{table}

Las pruebas de rendimiento de las diferentes implementaciones se han realizado en la equipo A. La
\refTab{TiemposGruposA} muestra los resultados de ejecutar el algoritmo con distintos n�meros de
trupo para cada una de las implementaciones. Para que los datos sean m�s fiables se han realizado
m�s de una repetici�n por configuraci�n. 

Entre los datos obtenidos hay que destacar que el algoritmo m�s r�pido es el ejecutado en la CPU.
Una de las causas es la poca potencia del dispositivo gr�fico con respecto de la CPU. 

Analizando los tiempos paralelos, lo primero que se puede observar son los resultados es como la
implementaci�n 0, la cual realiza la ejecuci�n del algoritmo sin paralelizar en la GPU. Esta
implementaci�n deja de ser viable en seguida obteniendo resultados superiores al minutos con 1.000
datos y 100 repeticiones. Los resultados obtenidos en la implementaci�n A, que paraleliza los
subconjuntos de elementos en cada fase, mejoran los tiempos de la implementaci�n 0, aunque con
10.000 elementos no los resultados dejan de ser �ptimos. 

La implementaci�n B y C, que obtienen tiempos muy similares son las primeras que obtienen tiempos
aceptables. Esta obtiene los mejores tiempos porque incluye
una nueva paralelizaci�n que consiste en dividir la organizaci�n de cada subconjunto. Esta
implementaci�n sin embargo no ordena completamente el vector, ya que no existe comunicaci�n entre
los hilos que ordenan el mismo subconjunto de elementos. La implementaci�n representa el mayor
grado de paralelizaci�n ya que aparentemente no es posible dividir m�s el trabajo de cada hilo.

La implementaci�n D realiza las mejoras necesarias para que la implementaci�n B ordene realmente
los elementos. Esta implementaci�n es aparentemente m�s lenta que la B sin embargo la diferencia de
tiempos entre ellas disminuye al aumentar el n�mero de elementos, llegando a conseguir mejores
tiempos al ordenar vectores de un mill�n de elementos. Esto puede deberse a que en cada fase, la
implementaci�n D deja m�s ordenado el vector. 

Otra observaci�n que hay que hacer es que la relaci�n entre el tiempo que tarda en acabar la
implementaci�n lineal y la D es menor cuanto m�s elementos tiene el vector. 

Las �ltimas implementaciones, E y F obtienen tambi�n tiempos similares entre ellos. Estas realizan
copias a memoria de cache para intentar mejorar los tiempos de ejecuci�n. Sin embargo los tiempos
no mejoran con respecto a las implementaciones anteriores. Esto posiblemente es debido a toda la
l�gica que a�ade la implementaci�n al realizar la copia. Entre los detalles de la implementaci�n
hay que se�alar que es necesario indicar en la implementaci�n OpenCL la cantidad de memoria local
que se va a necesitar y esto es uno de los problemas de estas implementaciones ya que cada thread
dispone de una cantidad m�xima de memoria que puede disponer para sus datos. De hecho la
implementaci�n no es capaz de ordenar m�s de un mill�n de elementos. 

% indicar en trabajo futuro que los hilos ordenen 4 o 5 elementos independientemente de cuantos
% tengan que manejar. 





La tabla \refTab{TiemposGruposA} muestra los resultados de distintas ejecuciones de las diferentes
implementaciones. Las implementaciones 0 y A muestran como al no estar paralelizado o tener una
paralelizaci�n bastante pobre, los resultados enseguida se disparan. Por otro lado los tiempos
nunca mejoran pero la relaci�n entre el tiempo de ejecuci�n lineal y el tiempo de ejecuci�n
paralela (Imp-D) es cada vez m�s peque�a. Hay que se�alar que la implementaci�n D consigue los
mejores tiempos en la �ltima ejecuci�n. La Implementaci�n E, que realiza copia en memoria local de
los datos del vector, obtiene tiempos inferiores a la implementaci�n D, esto se debe posiblemente
a la complejidad de la l�gica necesaria para controlar la copia local de los datos. 

Los tiempos de la Implementaci�n C no se han puesto por ser inferiores similares a la
implementaci�n B al igual que las implementaciones E y F. Tambi�n indicar que la implementaci�n E y
F es necesario definir manualmente en la implementaci�n OpenCL la cantidad de datos en local que va
a haber. 





\subsection{Numero de lecturas y escrituras}

\begin{table}
\begin{center}
\begin{tabular}{|l | c | c| c | c | c | c | }
\hline
\multicolumn{7}{|c|}{1000 x 1000 (Elementos x repeticiones) } \\
\hline
Incremento     & 929 & 505 & 209 & 109 & 41   & 19   \\
\hline
Hilos / grupo  &     1&    1 & 1    & 2    & 7    & 11 \\
Grupos         & 929  & 505  & 209  & 109  & 41   & 19 \\
Datos / hilo   & 1.07 & 1.98 & 4.48 & 4.58 & 3.48 & 4.78 \\
Lecturas       & 0.11 & 1.4  & 6.2  & 9.7  & 17   & 19 \\
Escritura      & 0.11 & 1.4  & 6.2  & 8.8  & 15   & 16 \\
Pasadas        & 1    & 1    & 1    & 1.6  & 2.5  & 2.8\\
\hline
\hline
\multicolumn{3}{|c|}{1000 x 1000} & ----- & \multicolumn{3}{c|}{10000 x 100}\\
\hline
Incremento     & 5   & 1     & ----- & 5    & 1   &  -----\\ 
\hline
Hilos / grupo  & 40  & 200   & ----- & 512  & 512  &  -----\\ 
Grupos         & 5   & 1     & ----- & 5    & 1    &  -----\\ 
Datos / hilo   & 5   & 5     & ----- & 3,9  & 19,5 &  -----\\ 
Lecturas       & 32  & 43    & ----- & 128  & 150  &  -----\\ 
Escritura      & 27  & 36    & ----- & 123  & 172  &  -----\\ 
Pasadas        & 4.0 & 5.5   & ----- & 4,1  & 22,6 &  -----\\ 
\hline
\end{tabular}
\caption{N�mero medio de lecturas, escrituras y pasadas hilo}
\labelTab{Medias}
\end{center}
\end{table}

En el apartado anterior se ha observado como realizar una copia local de los datos para realizar la
ordenaci�n no resulta efectiva. En este apartado se han analizado el n�mero de lecturas,
escrituras medias por hilo de la implementaci�n D y los resultados se muestran en la
\refTab{Medias}. 

El n�mero de pasadas indica la cantidad de veces que un algoritmo tiene que reordenar sus
elementos. 

Para 1.000 elementos, el n�mero de escrituras aumenta alcanzando m�s de 30 lecturas y escrituras en
la �ltima fase o lo que es lo mismo, aproximadamente 8 lecturas y 7 escrituras por dato en la �ltima
fase. Aunque no se un dato muy bueno se compensa con las pocas lecturas /
escrituras de las primeras fases del algoritmo. Este dato se debe a que cada vez hay m�s hilos por
subconjunto y los subconjuntos son m�s grandes. El n�mero medio total de lecturas y escrituras por
pasada y dato es aproximadamente 4 que, no es un valor muy negativo. El n�mero �ptimo de
lecturas y escrituras es aproximadamente 2. 

Para 10.000 elementos, se puede observar como en las dos ultimas fases, mientras la media de datos
por hilo no sea superior a 4, el n�mero de pasadas se mantiene en 4, sin embargo cuando se llega al
m�ximo de hilos por grupo de trabajo, 512, el n�mero de pasadas que hay que realizar tambi�n
aumenta, con estos datos se puede deducir que el n�mero de pasadas est� directamente relacionado
con el n�mero de datos que ordena cada hilo. Tambi�n se puede observar que la relaci�n entre el
n�mero de lecturas / escrituras por dato y el n�mero de hilos es, en la �ltima pasada, muy
inferior a la fase anterior. Este resultado te hace planea la duda de cual es la cantidad �ptima
datos que cada hilo debe ordenar y cuantos hilos como m�ximo deben lanzarse por subconjunto de
elementos en cada fase. En an�lisis de la siguiente secci�n terminar� se analizar�n la
diferencia de tiempos cambiando estos datos.

\subsection{An�lisis de los tiempos por fases}

% \begin{table}
% 
% \begin{tabular}{|l | c | c| c | c | c | c | }
% \hline
% \multicolumn{7}{|c|}{1000 x 1000 (Elementos x repeticiones) } \\
% \hline
% Incremento     & 929 & 505 & 209 & 109 & 41   & 19   \\
% \hline
% Hilos / grupo  &     1&    1 & 1    & 2    & 7    & 11 \\
% Grupos         & 929  & 505  & 209  & 109  & 41   & 19 \\
% Datos / hilo   & 1.07 & 1.98 & 4.48 & 4.58 & 3.48 & 4.78 \\
% Lecturas       & 0.11 & 1.4  & 6.2  & 9.7  & 17   & 19 \\
% Escritura      & 0.11 & 1.4  & 6.2  & 8.8  & 15   & 16 \\
% Pasadas        & 1    & 1    & 1    & 1.6  & 2.5  & 2.8\\
% \hline
% \hline
% \multicolumn{3}{|c|}{1000 x 1000} & ----- & \multicolumn{3}{c|}{10000 x 100}\\
% \hline
% Incremento     & 5   & 1     & ----- & 5    & 1   &  -----\\ 
% \hline
% Hilos / grupo  & 40  & 200   & ----- & 512  & 512  &  -----\\ 
% Grupos         & 5   & 1     & ----- & 5    & 1    &  -----\\ 
% Datos / hilo   & 5   & 5     & ----- & 3,9  & 19,5 &  -----\\ 
% Lecturas       & 32  & 43    & ----- & 128  & 150  &  -----\\ 
% Escritura      & 27  & 36    & ----- & 123  & 172  &  -----\\ 
% Pasadas        & 4.0 & 5.5   & ----- & 4,1  & 22,6 &  -----\\ 
% \hline
% \end{tabular}
% \caption{Tiempos de cada fase para un mill�n de elementos y 100 pasadas}
% \labelTab{TiemposXFases}
% \end{table}

\begin{table}
\begin{center}
\begin{tabular}{|l | c | c | c |}
\hline
\multicolumn{1}{|c}{Elementos}& \multicolumn{1}{c}{Repeticiones} & 
\multicolumn{1}{c}{Max. Hilos} & \multicolumn{1}{c|}{Min Elementos}\\
\multicolumn{1}{|c}{1.000.000}& \multicolumn{1}{c}{100} & 
\multicolumn{1}{c}{512} & \multicolumn{1}{c|}{1}\\
\hline
Incr.  & Lineal & Imp B & Imp D \\
\hline
587521 & 1.11  & 5.03  & 5.98  \\
260609 & 1.56  & 6.58  & 7.96  \\
146305 & 1.84  & 7.15  & 8.53  \\
64769  & 2.16  & 8.26  & 9.63  \\
36289  & 2.60  & 8.24  & 9.53  \\
16001  & 2.74  & 9.25  & 10.44 \\
8929   & 3.13  & 8.74  & 9.95  \\
3905   & 3.08  & 9.61  & 10.78 \\
2161   & 3.02  & 9.21  & 10.43 \\
929    & 3.18  & 9.90  & 11.24 \\
505    & 3.41  & 9.57  & 16.60 \\
209    & 3.71  & 6.10  & 11.48 \\
109    & 3.38  & 3.99  & 8.63  \\
41     & 7.61  & 5.36  & 8.45  \\
19     & 4.25  & 4.36  & 8.70  \\
5      & 3.77  & 20.97 & 11.66 \\
1      & 3.94  & 43.67 & 14.70 \\
\hline
Total  & 56.29 & 177.23 & 176.93 \\
\hline 
\hline
\multicolumn{1}{|c}{Elementos}& \multicolumn{1}{c}{Repeticiones} & 
\multicolumn{1}{c}{Max. Hilos} & \multicolumn{1}{c|}{Min Elementos}\\
\multicolumn{1}{|c}{1.000.000}& \multicolumn{1}{c}{100} & 
\multicolumn{1}{c}{256} & \multicolumn{1}{c|}{2}\\
\hline
Incr.  & Lineal & Imp B & Imp D \\
\hline
587521 & 1.17  & 5.07  & 5.81   \\
260609 & 1.57  & 6.56  & 7.98   \\
146305 & 1.83  & 7.06  & 8.40   \\
64769  & 2.20  & 8.46  & 9.73   \\
36289  & 2.55  & 8.17  & 9.41   \\
16001  & 2.79  & 9.37  & 10.41  \\
8929   & 3.14  & 8.50  & 9.93   \\
3905   & 3.09  & 9.82  & 10.87  \\
2161   & 3.03  & 9.06  & 10.41  \\
929    & 3.20  & 10.07 & 11.16  \\
505    & 3.39  & 9.55  & 10.73  \\
209    & 3.72  & 8.97  & 16.52  \\
109    & 3.46  & 5.84  & 11.32  \\
41     & 7.82  & 6.85  & 9.42   \\
19     & 4.33  & 4.93  & 8.48   \\
5      & 3.83  & 21.52 & 10.95  \\
1      & 3.89  & 16.74 & 13.62  \\
\hline
Total  & 56.83 & 157.66& 177.36 \\
\hline

\end{tabular}
\caption{Tiempo de ejecuci�n por fases}
\labelTab{FasesLineal}
\end{center}
\end{table}

\begin{table}
\begin{center}
\begin{tabular}{|l | c | c | c | c |}
\hline
\multicolumn{1}{|c}{Elementos}& \multicolumn{1}{c}{Repeticiones} & 
\multicolumn{1}{c}{Max. Hilos} & \multicolumn{2}{c|}{Min. Elementos} \\
\multicolumn{1}{|c}{500.000}& \multicolumn{1}{c}{100} & 
\multicolumn{1}{c}{512} & \multicolumn{2}{c|}{5} \\
\hline
Incr.  & Lineal & Imp B & Imp D & Imp E    \\
\hline 
260609 & 0.60   & 2.34   & 2.81   & 7.46   \\
146305 & 0.70   & 3.13   & 3.69   & 7.80   \\
64769  & 0.95   & 3.82   & 4.46   & 8.42   \\
36289  & 0.96   & 3.84   & 4.42   & 8.48   \\
16001  & 1.26   & 4.45   & 4.99   & 8.99   \\
8929   & 1.35   & 4.22   & 4.83   & 8.94   \\
3905   & 1.44   & 4.82   & 5.36   & 9.43   \\
2161   & 1.48   & 4.45   & 5.12   & 9.30   \\
929    & 1.56   & 4.95   & 5.52   & 9.74   \\
505    & 1.42   & 4.72   & 5.38   & 13.15  \\
209    & 1.64   & 5.47   & 5.96   & 7.60   \\
109    & 1.52   & 4.84   & 5.40   & 4.61   \\
41     & 2.88   & 5.16   & 10.46  & 3.06   \\
19     & 1.94   & 4.36   & 8.11   & 2.27   \\
5      & 1.87   & 14.18  & 9.48   & 1.45   \\
1      & 1.89   & 16.34  & 15.07  & 4.57   \\
\hline
Total  & 24.39 & 91.74   & 102.26 & 115.99 \\
\hline
\end{tabular}
\caption{Tiempo de ejecuci�n por fases}
\labelTab{FasesLineal2}
\end{center}
\end{table}

En la \refTab{FasesLineal} y la \refTab{FasesLineal2} se analizan los tiempos para cada una de
las fases. Los par�metros que
se comparan son el n�mero de hilos m�ximos por subconjunto y el n�mero m�nimo de elementos que
ordena cada hilo. Hay que se�alar que si el n�mero de elementos que tiene un
subconjunto es inferior m�nimo de datos por hilo, prevalece haber al menos un hilo por
subconjunto de elementos. 

Las pruebas se han realizado con dos configuraciones, la primera configuraci�n se realiza con el
mayor n�mero de elementos posibles para la implementaci�n D, un mill�n. Una segunda configuraci�n se
realiza con el mayor n�mero posible para la implementaci�n E, medio mill�n. 

Las pruebas tienen dos par�metros, el n�mero m�nimo de datos por hilo y el n�mero m�ximo de hilos
por subconjunto. El primero afecta principalmente a las primeras fases, valores
altos producen que estas fases tengan pocos hilos por subconjunto. Este valor deja de tener sentido
cuando la relaci�n entre el n�mero de datos de un subconjunto y el n�mero de hilos aumenta y es
esta relaci�n es la que indica cuantos datos va a ordenar cada hilo a partir de entonces. 

Con los resultados obtenidos, se puede determinar que el n�mero de datos m�nimo de datos por hilo no
afecta a la velocidad final mientras que no sea una cantidad muy grande.

Se puede observar tambi�n como la implementaci�n B ofrece valores m�s dispares como puede
verse en las ejecuciones donde solo cambia el par�metro de n�mero de datos m�nimo por hilo.

Tambi�n puede observarse en los resultados obtenidos como los tiempos de
ordenaci�n de cada una de las fases son constantes para la implementaci�n lineal y D, la
implementaci�n B pese a ser m�s r�pida que la D, empeora sensiblemente en las �ltimas fases. Una de
las causas puede ser el hecho de que los subconjuntos, despu�s de cada fase en la
implementaci�n B, no est�n realmente ordenados. 

Las ejecuciones en las que el par�metro m�ximo n�mero de hilos por subconjunto empeoran el
rendimiento en las �ltimas fases, en concreto con hasta 64 hilos por subconjunto. Con
GPUs que dispongan m�s n�cleos, este dato ser� posiblemente m�s decisivo. 

Por �ltimo hay, la configuraci�n con medio mill�n de datos, donde se puede ejecutar la
implementaci�n E, se puede observar como, al hacer una copia en cache de los datos, el rendimiento
del algoritmo en las fases finales del algoritmo es mejor debido a que se producen un mayor
n�mero de lecturas y escrituras por dato. 

 


 
% La 
% efectiva. Este hecho hace pensar que el n�mero de lecturas y escrituras no sea muy grande y, por
% tanto, no merezca la pena copiar los datos a local. 
% 
% La \refTab{Medias} muestra como el n�mero de accesos de lectura y escritura es cada vez mayor. En
% estas fases. Este incremento est� relacionado con las pasadas que hace cada hilo. Hay que
% aclarar que por pasada se entiende al n�mero de veces que ordena un hilo su subconjunto de
% datos, un hilo vuelve a ordenar sus datos si se ha producido un cambio entre los extremos de los
% datos que maneja cada hilo, al no encontrarse en la posici�n correcta. 

\subsection{Tiempos en distintos equipos}

\begin{table}
\begin{center}
\begin{tabular}{|l | c | c | c |}
\hline
& \multicolumn{3}{|c|}{Implementaci�n} \\
\hline
Equipo & Lineal   & B      & D     \\
\hline 
Equipo A & 4.05   & 11.59  & 13.24 \\
Equipo B & 4.53   & 7.80   & 7.64  \\
Equipo C & 3.92   & 5.61   & 3.78  \\
\hline
\end{tabular}
\caption{Tiempo de ejecuci�n en distintos equipos}
\labelTab{equipos}
\end{center}
\end{table}

Por �ltimos se han podido ejecutar tres implementaciones en tres equipos distintos, esto ha
permitido comparar con dispositivos gr�ficos m�s potentes. Las pruebas se han limitado a las
implementaciones Lineal, Shellsort B y Shellsort D. Las pruebas se han realizado con 400.000
elementos y 20 repeticiones. El n�mero m�ximo de hilos por subconjunto se ha establecido a 512 y
cada hilo ordena al menos 5 elementos. los resultados se encuentran en la \refTab{equipos}. 

La primera conclusi�n que puede obtenerse es que el algoritmo depende del n�mero de n�cleos que el
dispositivo posea. El dispositivo gr�fico del equipo B, sin ser m�s potente, posee el doble de
n�cleos y reduce a la mitad el tiempo de ejecuci�n del algoritmo. El equipo C posee 5 veces m�s
n�cleos y reduce el tiempo de ejecuci�n casi en una quinta parte. 

El otro dato positivo que se puede observar es que en la ejecuci�n de la implementaci�n D en el
equipo C mejora ligeramente el tiempo de la implementaci�n lineal. 



\section{Conclusi�n}



Los datos obtenidos en las ejecuciones de las diferentes implementaciones, sin ser concluyentes,
debido a la potencia de los dispositivos gr�ficos, hacen pensar las implementaciones paralelas
de este algoritmo pueden mejorar los tiempos de ordenaci�n con respecto a la ejecuci�n del mismo
en una CPU. Sin embargo esta mejora no es tan significativa como los resultados obtenidos con otros
algoritmos. 

Entre las causas se encuentra la deslocalizaci�n de los datos que gestiona cada hilo y la
diferencia de cantidad de datos que manejan cada subconjunto en las diferentes fases, sobre todo la
primera y la �ltima. 

Tambi�n se puede observar como, el algoritmo ofrece mejores resultados cuantos m�s datos gestiona. 

\section{Trabajo futuro} 

Entre las mejoras que se pueden realizar a la implementaci�n paralela plateada se encuentra realizar
una implementaci�n dependiente de los incrementos y diferentes implementaciones para cada
incremento. Al existir pocas fases, para un mill�n de elementos, 17 fases, la ganancia que se
obtendr� con esto, sera mucho mayor. Adem�s, esta implementaci�n permitir� que al copia en local de
los datos que cada hilo ordena, sea mucho m�s eficiente. 





\chapter{Lista de abreviaturas, acr�nimos y definiciones}
\begin{acronym}
 \acro{ACA}		{Arquitectura de componentes aut�nomos}
 \acro{ASPATH}	{AS\_PATH}	{Autonomous System Path}
 \acro{AS}		{Sistema Aut�nomo} %X
 \acrodefplural{AS}	[AS]	{Sistemas Aut�nomos} 
 \acro{BGP}	[BGP-4]	{Border Gateway Protocol} %ref4271
 \acro{BSD}		{Berkeley Software Distribution} %X
 \acro{CSL}		{Capa de Servicios de Red}
 \acro{DML}		{Domain Modeling Language}
 \acro{EBGP}	[eBGP]	{BGP externo} %X
 \acro{EGP}		{Protocolo de encaminamiento externo} %X
 \acro{IBGP}	[iBGP]	{BGP interno} %X
 \acro{IGP}		{Protocolo de enrutamiento interno} %X
 \acro{ISP}		{Proveedor de Internet} %X
 \acrodefplural{ISP}	[ISP's]	{Proveedores de Internet}
 \acro{JSIM}	[J-Sim]	{JavaSim} %X
 \acro{MED}		{Multi-Exit Discriminator}
 \acro{NLRI}		{Network Layer Reachability Information}
 \acro{OSPF}		{Open Shortest Path First} %X
 \acro{RIP}		{Routing Information Protocol} %X
 \acro{RIPE}		{R�seaux IP Europ�ens} %X	
 \acro{RIR}		{Registro Regional de Internet}
 \acro{RFC}		{Request For Comments} %X
 \acro{RR}		{Reflector de Rutas} %X
 \acro{RRC}		{Cliente de reflexi�n de rutas} %X
 \acro{TCL}	[Tcl]	{Tool Command Language}
 \acro{TK}	[Tk]	{Tool Kit}
 \acro{XML}		{Extensible Markup Language}
\end{acronym}


\appendices

\section{Equipos utilizados}

\subsection{EquipoA}

\subsubsection*{CPU} Intel Core i5 560M 
\subsubsection*{GPU} Nvidia Quadro NVS 3100M (16 N�cleos CUDA)
\subsubsection*{Memoria principal} 4 GB. 
\subsubsection*{Memoria Gr�fica} 512 MB. 
\subsubsection*{Sistema operativo} Ubuntu 12.04
\subsubsection*{OpenCL} 1.1 (Cuda 4.2.1)

\subsection{EquipoB}
\subsubsection*{CPU} Intel Core i5 650
\subsubsection*{GPU} Nvidia GT 320 (72 N�cleos CUDA)
\subsubsection*{Memoria principal} 4GB. 
\subsubsection*{Memoria Gr�fica} 512 MB.
\subsubsection*{Sistema operativo} Debian 6 
\subsubsection*{OpenCL} OpenCL 1.1 (Cuda 4.0) 

\subsection{EquipoC}
\subsubsection*{CPU} Intel Core 2 Duo 
\subsubsection*{GPU} Nvidia GeForce 8600 GT (32 N�cleos CUDA) 
\subsubsection*{Memoria principal} 2GB
\subsubsection*{Memoria Gr�fica} 256 Mb
\subsubsection*{Sistema operativo} Debian 6
\subsubsection*{OpenCL} 1.1 (Cuda 4.0)



% \input{Apendices/BApendix}
% \section{Proof of the First Zonklar Equation}
% Appendix one text goes here.
% 
% % you can choose not to have a title for an appendix
% % if you want by leaving the argument blank
% \section{Apendix two}
% Appendix two text goes here.


% use section* for acknowledgement
% Can use something like this to put references on a page
% by themselves when using endfloat and the captionsoff option.



% trigger a \newpage just before the given reference
% number - used to balance the columns on the last page
% adjust value as needed - may need to be readjusted if
% the document is modified later
%\IEEEtriggeratref{8}
% The "triggered" command can be changed if desired:
%\IEEEtriggercmd{\enlargethispage{-5in}}

% references section

% can use a bibliography generated by BibTeX as a .bbl file
% BibTeX documentation can be easily obtained at:
% http://www.ctan.org/tex-archive/biblio/bibtex/contrib/doc/
% The IEEEtran BibTeX style support page is at:
% http://www.michaelshell.org/tex/ieeetran/bibtex/
%\bibliographystyle{IEEEtran}
% argument is your BibTeX string definitions and bibliography database(s)
%\bibliography{IEEEabrv,../bib/paper}
%
% <OR> manually copy in the resultant .bbl file
% set second argument of \begin to the number of references
% (used to reserve space for the reference number labels box)
% \bibliography{rfc,myBib}
% \bibliography{}
% \bibliographystyle{abbrv}

% biography section
% 
% If you have an EPS/PDF photo (graphicx package needed) extra braces are
% needed around the contents of the optional argument to biography to prevent
% the LaTeX parser from getting confused when it sees the complicated
% \includegraphics command within an optional argument. (You could create
% your own custom macro containing the \includegraphics command to make things
% simpler here.)
%\begin{biography}[{\includegraphics[width=1in,height=1.25in,clip,keepaspectratio]{mshell}}]{Michael Shell}
% or if you just want to reserve a space for a photo:


% if you will not have a photo at all:

% insert where needed to balance the two columns on the last page with
% biographies
%\newpage


% You can push biographies down or up by placing
% a \vfill before or after them. The appropriate
% use of \vfill depends on what kind of text is
% on the last page and whether or not the columns
% are being equalized.

%\vfill

% Can be used to pull up biographies so that the bottom of the last one
% is flush with the other column.
%\enlargethispage{-5in}



% that's all folks
\end{document}

