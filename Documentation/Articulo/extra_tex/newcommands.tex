\newcommand{\labelEq}[1]{\label{eq:#1}}
\newcommand{\labelFig}[1]{\label{fig:#1}}
\newcommand{\labelTab}[1]{\label{tab:#1}}
\newcommand{\labelSec}[1]{\label{sec:#1}}
\newcommand{\refEq}[1]{Ecuaci�n~\eqref{eq:#1}}
\newcommand{\refFig}[1]{Fig.\ref{fig:#1}}
\newcommand{\refTab}[1]{tabla~\ref{tab:#1}}
\newcommand{\RefTab}[1]{Tabla~\ref{tab:#1}}
\newcommand{\refSec}[1]{secci�n~\ref{sec:#1}}
\newcommand{\RefSec}[1]{Secci�n~\ref{sec:#1}}
\newcommand{\refApe}[1]{ap�ndice~\ref{sec:#1}}
\newcommand{\RefApe}[1]{Ap�ndice~\ref{sec:#1}}
\newcommand{\angl}[1]{{\em #1}}
\newcommand{\cod}[1]{{\tt #1}}

%% I have got rid of the .eps file
%% extension in the filename, in order to allow PDFLATEX to
%% import the PDF's, which I have also created and included
%% in the zip file.
%% I normally use this macro for this:
%% \newFig{filename}{Caption}
%% I use the <filename> as the label... it's easier to
%% remember for me...





\ifpdf
\newcommand{\newFig}[2] {
\begin{figure}[ht]
\centering
\includegraphics [width=0.40\textwidth] {Figs/pdf/#1}
\caption {#2}
\labelFig{#1}
\end{figure}
}

\newcommand{\newFigWidth}[3] {
\begin{figure}[ht]
\centering
\includegraphics [width=#1\textwidth] {Figs/pdf/#2}
\caption {#3}
\labelFig{#2}
\end{figure}
}


\newcommand{\newFigTwoXTree}[8] {
\begin{figure}[ht!]
  \centering
   %%----1x1----
  \subfloat[]{
        \labelFig{#3}         %% Etiqueta para la primera subfigura
        \includegraphics[width=0.21\textwidth]{Figs/pdf/#3}}
   %%----1x2----
  \subfloat[]{
        \labelFig{#4}         %% Etiqueta para la segunda subfigura
        \includegraphics[width=0.21\textwidth]{Figs/pdf/#4}}\\[20pt]
   %%----2x1----
  \subfloat[]{
        \labelFig{#5}         %% Etiqueta para la primera subfigura
        \includegraphics[width=0.21\textwidth]{Figs/pdf/#5}}
   %%----2x2----
  \subfloat[]{
        \labelFig{#6}         %% Etiqueta para la segunda subfigura
        \includegraphics[width=0.21\textwidth]{Figs/pdf/#6}}\\[20pt]
   %%----3x1----
  \subfloat[]{
        \labelFig{#7}         %% Etiqueta para la primera subfigura
        \includegraphics[width=0.21\textwidth]{Figs/pdf/#7}}
   %%----3x2----
  \subfloat[]{
        \labelFig{#8}         %% Etiqueta para la segunda subfigura
        \includegraphics[width=0.21\textwidth]{Figs/pdf/#8}}
  \labelFig{#1}                %% Etiqueta para la figura entera
  \caption{#2}
\end{figure}
}







\else
\newcommand{\newFig}[2] {
\begin{figure}[ht]
\centering
\includegraphics [width=0.40\textwidth] {Figs/eps/#1}
\caption {#2}
\labelFig{#1}
\end{figure}
}

\newcommand{\newFigWidth}[3] {
\begin{figure}[ht]
\centering
\includegraphics [width=#1\textwidth] {Figs/eps/#2}
\caption {#3}
\labelFig{#2}
\end{figure}
}
\fi


